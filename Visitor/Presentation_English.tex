\documentclass{beamer}

\usepackage{fontspec}
\usepackage{minted}
\usepackage{forest}

\setmainfont{CMU Serif}
\setsansfont{CMU Sans Serif}
\setmonofont{CMU Typewriter Text}

\usetheme{Madrid}
\usecolortheme{default}

\title[Implementation of visitor pattern in C++]{Implementation of visitor pattern in C++ by using std::variant}
\author[Popov, Karpenko]{Popov Ruslan (first-year student, group КІ-23-1), Karpenko Nadiia}
\institute[DNU]{Faculty of Physics, Electronics and Computer Systems, Oles Honchar Dnipro National University, Haharina Ave, 72, Dnipro}
\date[MEICS-2023]{MEICS, November 2023}

\begin{document}
    \begin{frame}
        \titlepage
    \end{frame}

    \begin{frame}[fragile]{Abstract syntax tree}
   \begin{exampleblock}{The source code:}
\begin{minted}[fontsize=\small, tabsize=4]{c}
int main()
{
    printf("hello world");
    return 0;
}
\end{minted}
\end{exampleblock}

   \begin{block}{Parsing result (simplified):}
                \begin{center}
                    \begin{forest}
                        /tikz/every node/.append style={font=\footnotesize},
                        [func\_decl
                            [int ]
                            [main ]
                            [block\_stmt
                                    [call\_expr
                                        [printf ]
                                        ["hello world" ] ]
                                [return\_stmt 
                                    [0 ] ] ] ] ]	
                    \end{forest}
                \end{center}
    \end{block}

    \end{frame}

    \begin{frame}{Functionality over AST}
        \begin{block}{print$: AST \to String$}
            print $Var(a)$ = $"a"$\\
            print $UnaryOp(-, Var(b))$ = $"-b"$\\
        \end{block}
    
        \begin{exampleblock}{eval$: AST \to Value$}
            eval $Integer(7)$ = $7$\\
            eval $BinOp(Integer(2), +, Integer(2))$ = $4$\\
        \end{exampleblock}
    
        \begin{alertblock}{optimize$: AST \to AST$}
            optimize $BinOp(Var(a), *, 4)$ = $BinOp(Var(a), <<, 2)$\\
        \end{alertblock}
    \end{frame}

    \begin{frame}[fragile]{First attempt to solve expression problem}

 \begin{alertblock}{Node interface}
\begin{minted}[fontsize=\small, tabsize=4]{c++}
struct Node {
    virtual ~Node() {}
    virtual std::string ToString() = 0;
    virtual int Eval() = 0;
    virtual Node* Optimize() = 0;
};
\end{minted}
 \end{alertblock}
\begin{block}{Binary expression node}
\begin{minted}[fontsize=\small, tabsize=4]{c++}
struct BinOp : Node {
    Expr* left, right;
    BinOpType op;
    ...
    virtual std::string ToString();
    virtual int Eval();
    virtual Node* Optimize();
};
\end{minted}
\end{block}
    \end{frame}

    \begin{frame}[fragile]{Classical approach: Visitor pattern}
        \begin{center}
            \begin{forest}
                [Expression problem
                [AST node as a class
                    (\textit{data}) ]
                [\underline{Functionality as a \textbf{class}}
                    (\textit{code}) ] ]
            \end{forest}
        \end{center}
        
        \begin{columns}
            \begin{column}{0.47\textwidth}
                \begin{exampleblock}{\small An AST node}
\begin{minted}[fontsize=\tiny, tabsize=4]{c++}
struct BinOp : Node {
    Expr* left, right;
    BinOpType op;
    
    virtual void Accept(Visitor* vis) {
        vis->VisitBinOp(this);	
    }
};
\end{minted}
                \end{exampleblock}
            \end{column}
        
            \begin{column}{0.47\textwidth}
                \begin{block}{\small Visitor}
\begin{minted}[fontsize=\tiny, tabsize=4]{c++}
struct Visitor {
    virtual void VisitInteger(Integer* expr) = 0;
    virtual void VisitUnaryOp(UnaryOp* expr) = 0;
    virtual void VisitBinOp(BinOp* expr) = 0;
}
\end{minted}
                \end{block}
            \end{column}
        \end{columns}
    \end{frame}

    \begin{frame}{Problems with Visitor pattern}
        \begin{itemize}
            \item Not so easy to understand.
            \item A lot of things to do in order to add a new AST node:
            \begin{enumerate}
                \item Create a new method \texttt{Visit} to \texttt{Visitor} interface.
                \item Create a new method \texttt{Accept} in the new AST node.
            \end{enumerate}
        \end{itemize}
        
        \vspace{30pt}
        
        \begin{block}{A question}
            Can we implement this pattern easier?
        \end{block}
    \end{frame}

    \begin{frame}[fragile]{\texttt{std::variant} in C++}

        \begin{columns}
            \begin{column}{0.45\textwidth}
\begin{block}{Until C++17}
\begin{minted}[tabsize=4]{c++}
// A simplified example.
struct SumType {
    int AsInteger();
    bool IsInteger();
    
    ValueType GetType();
    
private:
    ValueType type;
    union {
        int num;
        ...
    } as;
};
\end{minted}
\end{block}
            \end{column}

            \begin{column}{0.5\textwidth}
\begin{block}{Since C++17}
\begin{minted}[tabsize=4]{c++}
#include <variant>

// Greatly reduced code size.
using SumType =
    std::variant<int, ...>;
\end{minted}
\end{block}
            \end{column}
        \end{columns}
    \end{frame}
    
    \begin{frame}[fragile]{\texttt{std::visit} у C++}
        \begin{block}{Declaration of \texttt{std::visit} since C++17}
\begin{minted}[fontsize=\small, tabsize=4]{c++}
template <class Visitor, class... Variants>
constexpr /* ... */ visit(Visitor&& vis, Variants&&... vars);
\end{minted}
        \end{block}

        \begin{center}
            Usage example:
        \end{center}
\begin{minted}[tabsize=4]{c}
struct MyVisitor {
    std::string operator()(int arg)   { return "integer"; }
    std::string operator()(bool arg)  { return "boolean"; }
};

int main() {
    std::variant<int, bool> var = 10;
    MyVisitor vis;
    std::cout << std::visit(vis, var) << std::endl;
    return 0;
}
\end{minted}
    \end{frame}

    \begin{frame}[fragile]{Our approach to expression problem}
        \begin{block}{Represent abstract AST nodes as variants}
\begin{minted}{c++}
// A greately simplified example.
using Expr = std::variant<Int, Unary, Binary, ...>;
using Stmt = std::variant<If, While, Return, ...>;
\end{minted}
        \end{block}
\begin{minted}[tabsize=4]{c}
struct Printer {
    std::string operator()(Int& expr);
    ...
    std::string operator()(If& stmt);
    ...
    std::string PrintExpr(Expr* expr) {
        return std::visit(*this, *expr);
    }
};
\end{minted}
    \end{frame}

    \begin{frame}{Some problems with our approach}
        \begin{itemize}
            \item \texttt{std::variant} is available since C++17 $\to$ earlier standards are not supported.
            \item \texttt{std::variant} uses template metaprogramming $\to$ complex error reports.
            \item Do not use an \texttt{Expr} inside an \texttt{Expr} or a \texttt{Stmt} inside a \texttt{Stmt} directly $\to$ structure inside itself.
        \end{itemize}
    \end{frame}

    \begin{frame}{End of the report}
        \begin{center}
            \huge Conclusions
        \end{center}
    \end{frame}

    \begin{frame}{End of the report}
        \begin{center}
            \huge Thanks for attention!
        \end{center}
        
        \begin{exampleblock}{Additional information}
            Source code: \textcolor{blue}{\href{https://github.com/InAnYan/AstVisitor}{https://github.com/InAnYan/AstVisitor}}.
            
            Email:  \textcolor{blue}{ \href{mailto:popov_ro@fffeks.dnu.edu.ua}{popov\_ro@fffeks.dnu.edu.ua}}.
        \end{exampleblock}
        
        \begin{block}{Authors}
            \textbf{Popov Ruslan (first-year student, group КІ-23-1), Karpenko Nadiia}.
            
            \textit{Faculty of Physics, Electronics and Computer Systems, Oles Honchar Dnipro National University, Haharina Ave, 72, Dnipro}
        \end{block}
    \end{frame}
\end{document}

