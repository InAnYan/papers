\documentclass{beamer}

\usepackage{fontspec}
\usepackage{minted}
\usepackage{forest}

\setmainfont{CMU Serif}
\setsansfont{CMU Sans Serif}
\setmonofont{CMU Typewriter Text}

\usetheme{Madrid}
\usecolortheme{default}

\title[Реалізація патерну Visitor у мові C++]{Реалізація патерну Visitor у мові C++ за допомогою std::variant}
\author[Попов, Карпенко]{Попов Руслан (студент 1 курсу, гр. КІ-23-1), Карпенко Надія}
\institute[ДНУ]{Факультет фізики, електроніки та комп’ютерних систем, Дніпровський національний університет імені Олеся Гончара, проспект Гагаріна, 72, м. Дніпро}
\date[MEICS-2023]{MEICS, November 2023}

\begin{document}
	\begin{frame}
		\titlepage
	\end{frame}

	\begin{frame}[fragile]{Дерево абстратного синтаксису}
   \begin{exampleblock}{Вихідний код:}
\begin{minted}[fontsize=\small, tabsize=4]{c}
int main()
{
    printf("hello world");
    return 0;
}
\end{minted}
\end{exampleblock}

   \begin{block}{Результат парсингу (спрощено):}
				\begin{center}
					\begin{forest}
						/tikz/every node/.append style={font=\footnotesize},
						[func\_decl
                            [int ]
							[main ]
							[block\_stmt
									[call\_expr
										[printf ]
										["hello world" ] ]
								[return\_stmt 
									[0 ] ] ] ] ]	
					\end{forest}
				\end{center}
    \end{block}

	\end{frame}

	\begin{frame}{Функції над ДАС}
		\begin{block}{print$: AST \to String$}
			print $Var(a)$ = $"a"$\\
			print $UnaryOp(-, Var(b))$ = $"-b"$\\
		\end{block}
	
		\begin{exampleblock}{eval$: AST \to Value$}
			eval $Integer(7)$ = $7$\\
			eval $BinOp(Integer(2), +, Integer(2))$ = $4$\\
		\end{exampleblock}
	
		\begin{alertblock}{optimize$: AST \to AST$}
			optimize $BinOp(Var(a), *, 4)$ = $BinOp(Var(a), <<, 2)$\\
		\end{alertblock}
	\end{frame}

	\begin{frame}[fragile]{Перша спроба розв'язання expression problem}

 \begin{alertblock}{Інтерфейс Node}
\begin{minted}[fontsize=\small, tabsize=4]{c++}
struct Node {
    virtual ~Node() {}
    virtual std::string ToString() = 0;
	virtual int Eval() = 0;
	virtual Node* Optimize() = 0;
};
\end{minted}
 \end{alertblock}
\begin{block}{Бінарний вираз}
\begin{minted}[fontsize=\small, tabsize=4]{c++}
struct BinOp : Node {
	Expr* left, right;
	BinOpType op;
	...
	virtual std::string ToString();
	virtual int Eval();
	virtual Node* Optimize();
};
\end{minted}
\end{block}
        %\begin{columns}
        %    \begin{column}{0.5\textwidth}
%
        %    \end{column}
        %    \begin{column}{0.5\textwidth}
%
       %     \end{column}
       % \end{columns}

	\end{frame}

	\begin{frame}[fragile]{Класичний підхід: патерн Visitor}
		\begin{center}
			\begin{forest}
				[Expression problem
				[ДАС як клас
					(\textit{дані}) ]
				[\underline{Функції як \textbf{клас}}
					(\textit{код}) ] ]
			\end{forest}
		\end{center}
		
		\begin{columns}
			\begin{column}{0.47\textwidth}
				\begin{exampleblock}{\small Вузол ДАС}
\begin{minted}[fontsize=\tiny, tabsize=4]{c++}
struct BinOp : Node {
	Expr* left, right;
	BinOpType op;

	virtual void Accept(Visitor* vis) {
		vis->VisitBinOp(this);	
	}
};
\end{minted}
				\end{exampleblock}
			\end{column}
		
			\begin{column}{0.47\textwidth}
				\begin{block}{\small Visitor}
\begin{minted}[fontsize=\tiny, tabsize=4]{c++}
struct Visitor {
	virtual void VisitInteger(Integer* expr) = 0;
	virtual void VisitUnaryOp(UnaryOp* expr) = 0;
	virtual void VisitBinOp(BinOp* expr) = 0;
}
\end{minted}
				\end{block}
			\end{column}
		\end{columns}
	\end{frame}

	\begin{frame}{Проблеми з патерном Visitor}
		\begin{itemize}
			\item Не легкий у розумінні.
			\item Треба робити багато речей для додавання нового вузла ДАС:
			\begin{enumerate}
				\item Додати метод \texttt{Visit} до інтерфейсу \texttt{Visitor}.
				\item Створити метод \texttt{Accept} у новому вузлі ДАС.
			\end{enumerate}
		\end{itemize}
		
		\vspace{30pt}
		
		\begin{block}{Питання}
			Чи можна реалізувати цей патерн легше?
		\end{block}
	\end{frame}

	\begin{frame}[fragile]{\texttt{std::variant} у C++}

		\begin{columns}
			\begin{column}{0.45\textwidth}
\begin{block}{До C++17}
\begin{minted}[tabsize=4]{c++}
// A simplified example.
struct SumType {
	int AsInteger();
	bool IsInteger();
	
	ValueType GetType();
	
private:
	ValueType type;
	union {
		int num;
		...	
	} as;
};
\end{minted}
\end{block}
			\end{column}

			\begin{column}{0.5\textwidth}
\begin{block}{З C++17}
\begin{minted}[tabsize=4]{c++}
#include <variant>

// Greatly reduced code size.
using SumType =
	std::variant<int, ...>;
\end{minted}
\end{block}
			\end{column}
		\end{columns}
	\end{frame}
	
	\begin{frame}[fragile]{\texttt{std::visit} у C++}
		\begin{block}{Декларація \texttt{std::visit} починаючи з C++17}
\begin{minted}[fontsize=\small, tabsize=4]{c++}
template <class Visitor, class... Variants>
constexpr /* ... */ visit(Visitor&& vis, Variants&&... vars);
\end{minted}
		\end{block}

		\begin{center}
			Приклад використання:
		\end{center}
\begin{minted}[tabsize=4]{c}
struct MyVisitor {
	std::string operator()(int arg)   { return "integer"; }
	std::string operator()(bool arg)  { return "boolean"; }
};

int main() {
	std::variant<int, bool> var = 10;
	MyVisitor vis;
	std::cout << std::visit(vis, var) << std::endl;
	return 0;
}
\end{minted}
	\end{frame}

	\begin{frame}[fragile]{Наш підхід до expression problem}
		\begin{block}{Представити абстрактні вузли ДАС як варіанти}
\begin{minted}{c++}
// A greately simplified example.
using Expr = std::variant<Int, Unary, Binary, ...>;
using Stmt = std::variant<If, While, Return, ...>;
\end{minted}
		\end{block}
\begin{minted}[tabsize=4]{c}
struct Printer {
	std::string operator()(Int& expr);
	...
	std::string operator()(If& stmt);
	...
	std::string PrintExpr(Expr* expr) {
		return std::visit(*this, *expr);
	}
};
\end{minted}
	\end{frame}

	\begin{frame}{Деякі недоліки нашого способу}
		\begin{itemize}
			\item \texttt{std::variant} доступний з C++17 $\to$ ранні стандарти не підтримуються.
			\item \texttt{std::variant} використовує шаблонне метапрограмування $\to$ складні повідомлення про помилки компіляції.
			\item Не використовуйте \texttt{Expr} всередені \texttt{Expr} або \texttt{Stmt} всередені \texttt{Stmt} $\to$ структура усередені самої себе.
		\end{itemize}
	\end{frame}

    \begin{frame}{Кінець доповіді}
        \begin{center}
            \huge Висновки
        \end{center}
    \end{frame}

	\begin{frame}{Кінець доповіді}
		\begin{center}
			\huge Дякую за увагу!
		\end{center}
		
		\begin{exampleblock}{Додаткова інформація}
			Детальний вихідний код: \textcolor{blue}{\href{https://github.com/InAnYan/AstVisitor}{https://github.com/InAnYan/AstVisitor}}.
			
			Електронна пошта:  \textcolor{blue}{ \href{mailto:popov_ro@fffeks.dnu.edu.ua}{popov\_ro@fffeks.dnu.edu.ua}}.
		\end{exampleblock}
		
		\begin{block}{Автори}
			\textbf{Попов Руслан (студент 1 курсу, гр. КІ-23-1), Карпенко Надія}.
			
			\textit{Факультет фізики, електроніки та комп’ютерних систем, Дніпровський національний університет імені Олеся Гончара, проспект Гагаріна, 72, м. Дніпро}
		\end{block}
	\end{frame}
\end{document}